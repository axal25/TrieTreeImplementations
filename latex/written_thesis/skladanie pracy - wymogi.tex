1. oświadczenia po pl i ang drukować dwustronnie

2. wykaz praktyk i osiągnięć studenta to po prostu dokument, na którym ma być imię i nazwisko, nr albumu, kierunek studiów i wypisane praktyki i osiągnięcia, które chcemy mieć umieszczone na dyplomie, w przypadku gdy chcemy odpis po ang to musimy również na tym wykazie wszystko sobie przetłumaczyć i napisać pod spodem po ang
    
3. w przypadku gdy chcemy odpis dyplomu po ang to poza wnioskiem dostępnym na stronie wydziału musimy też wydrukować wniosek z wirtualnego dziekanatu -> Dyplom -> Wniosek odpisy j. ang. umowa 

4. należy podpisać oba wydruki pracy (w twardej oprawie i w wersji 4 strony na jednej), np. pod oświadczeniem, że wykonało się pracę samodzielnie 

5. dokumenty umieścić w białej teczce podpisanej imieniem i nazwiskiem, nr.indeksu i kierunek.

6. fontu nikt nie sprawdza więc spokojnie można mieć domyślny 11pt z szablonu, nie trzeba nic kombinować; ktoś tam sieje ferment i jest zamieszanie a tak naprawdę te wytyczne są tylko po to żeby nikt nie napisał comic sansem rozmiar 24

7. tak samo z tymi polskimi znakami, wygląda dobrze w dokumencie to wystarczy, a że to są osobne litery i ogonki to już nie nasza wina

8. i ogółem antyplagiat jest bez sensu bo i tak jak tam wychodzą jakieś głupoty to zaznaczają że się nie liczy, no ale jest jakaś durna ustawa że trzeba używać więc trudno

9. zdjęcia (5) z dokumentami oddawanymi do dziekanatu, razem z tymi oświadczeniami co się tylko drukuje i daje w formie fizycznej. kolorowe zdjęcia formatu 4,5 × 6,5 (w stroju oficjalnym), w tym jedno opisane z tyłu imieniem, nazwiskiem i nr PESEL

10. na stronie tytułowej pracy powinno być napisane: "projekt dyplomowy" a NIE PRACA dyplomowa

11. Streszczenie projektu lub pracy dyplomowej w języku polskim i angielskim wg. wzoru w Załączniku 4. Czy ma być w treści pracy czy oddzielnie?


\href{
https://docs.google.com/document/d/1teuj2QOWyLMdtm7R-8wHDmPq2z-jhPZ2pBM2T-85Rhw/edit?fbclid=IwAR2DIe5W2CLKaIuucfKfYNmK2Qp7C3ucZizeC_ChH143hc8uECGxxEkI5N0#heading=h.b7lg7y1i0an9
}{uwagi do pracy}

\href{
https://docs.google.com/document/d/13Z51N8-IKl-xCPCpmJH0JaL9-bdoPklg4x4-XxaClgE/edit?fbclid=IwAR20b3PNkYSgeDMm6de2ZYqtGiTEUl9yZnfmoCo-IeSdBiyrsEvR--ysZRw
}{pytania obrona}

\href{
http://home.agh.edu.pl/~pszwed/wiki/doku.php?id=tematy_prac_inzynierskich
}{Piotr Szwed - Tematy prac inżynierski i proces dyplomowania}